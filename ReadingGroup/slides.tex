\documentclass{beamer}
%\documentclass[handout,t]{beamer}

\batchmode
% \usepackage{pgfpages}
% \pgfpagesuselayout{4 on 1}[letterpaper,landscape,border shrink=5mm]

\usepackage{amsmath,amssymb,amsfonts,enumerate,epsfig,bbm,calc,color,ifthen,capt-of}

\usetheme{Berlin}
\usecolortheme{mit}
\usefonttheme[onlymath]{serif}

\title{Task-agnostic Continual Learning with Hybrid Probabilistic Models}
\author[Polina Kirichenko \and Mehrdad Farajtabar]{
  Polina Kirichenko \inst{1} \and Mehrdad Farajtabar \inst{2} \and Dushyant Rao \inst{2} \\
  Balaji Lakshminarayanan \inst{3} \and Nir Levine \inst{2} \and Ang Li \inst{2} \\
  Huiyi Hu \inst{2} \and Andrew Gordon Wilson \inst{1} \and Razvan Pascanu \inst{2}
}

\institute[NYU \and DeepMind \and Google Brain]{
  \inst{1} New York University \and
  \inst{2} DeepMind \and
  \inst{3} Google Brain
}
\date{\today}
% \pgfdeclareimage[height=0.5cm]{mit-logo}{mit-logo.pdf}
% \logo{\pgfuseimage{mit-logo}\hspace*{0.3cm}}

% \AtBeginSection[]
% {
%   \begin{frame}<beamer>
%     \frametitle{Outline}
%     \tableofcontents[currentsection]
%   \end{frame}
% }
\beamerdefaultoverlayspecification{<+->}
% -----------------------------------------------------------------------------
\begin{document}
% -----------------------------------------------------------------------------

\frame{\titlepage}

\section[Outline]{}
\begin{frame}{Outline}
  \tableofcontents
\end{frame}

% -----------------------------------------------------------------------------
\section{Introduction}
% \subsection{Smoot's Ear: The Measure of Humanity}
\begin{frame}{Existing Approaches}
  \begin{itemize}
    \item<1-> re-sample the data or design specific loss functions that better facilitate learning with imbalanced data
    \item<1-> enhance recognition performance of the tail classes by transferring knowledge from the head classes
  \end{itemize}
  
  \begin{block}{Title of}
    hi
  \end{block}
\end{frame}
\begin{frame}{Current Situation}
  \begin{itemize}
    \item<1-> From a probability theory perspective, it is unjustifiable to use single point-estimates as weights to base any classification on.
    \item<1-> Modelling a distribution over the kernels (also known as filters) of a Cnns has never been attempted successfully before.
    \item<1-> The variational approach used to approximate the posterior in Bayesian NNs can be fairly computationally expensive – the use of Gaussian approximating distributions increases the number of model parameters considerably, without increasing model capacity by much.
  \end{itemize}
\end{frame}
\begin{frame}{Our Hypothesis}
  we approximate the intractable true posterior probability distributions $p(w|\mathcal{D})$
  with variational probability distributions $q_\theta(w|\mathcal{D})$,
  which comprise the properties of Gaussian distributions $\mu \in \Re^d$ and $\sigma \in \Re^d$,
  denoted by $\mathcal{N}(\theta|\mu, \sigma^2)$, where $d$ is the total number of parameters defining a probability distribution. 
\end{frame}
\begin{frame}{Our Contribution (1)}
  \begin{itemize}
    \item<1-> Present how \textit{Bayes by Backprop} can be efficiently applied to Cnns.
    \item<1-> Show how the model learns richer representations and predictions from cheap model
    averaging.
    \item<1-> Show that the proposed variational inference method for Bayesian Cnns can be applied to various Cnn architectures without any limitations
    on their performances.
  \end{itemize}
\end{frame}
\begin{frame}{Our Contribution (2)}
  \begin{itemize}
    \item<1-> Examine how to estimate the aleatoric and epistemic uncertainties and empirically
    show how the uncertainty can decrease, allowing the decisions made by the network to
    become more deterministic as the training accuracy increases.
    \item<1-> Show how the method typically only doubles the number of parameters yet trains an infinite ensemble using unbiased Monte Carlo estimates of the
    gradients.
    \item<1-> Apply L1 norm to the trained model parameters and prune the number of non
    zero values.
    \item<1-> Apply the concept of Bayesian CNN to tasks like Image Super-Resolution
    and Generative Adversarial.
  \end{itemize}
\end{frame}

% -----------------------------------------------------------------------------
\section{Background and Notation}
\begin{frame}{Neural Networks}
  \begin{itemize}
    \item<1-> Brain Analogies
    \item<1-> Neural Network
    \item<1-> Convolutional Neural Network
  \end{itemize}
\end{frame}
\begin{frame}{Probabilistic Machine Learning (1)}
  \begin{equation}
    p(y^*|x^*, X, Y) = \int p(y^*|f^*)p(f^*|x^*, X, Y) df^*
  \end{equation}
  The equation is intractable due to the integration sign. We can approximate it by taking a finite set of random variables $w$ and conditioning the model on it.
  \[
    p(y^*|x^*, X, Y) = \int p(y^*|f^*)p(f^*|x^*, w)p(w|X, Y) df^*dw
  \]
\end{frame}
\begin{frame}{Probabilistic Machine Learning (2)}
  we approximate $p(w|X, Y)$ with a variational distribution $q(w)$. Then
  \begin{equation}
    q(y^*|x^*) = \int p(y^*|f^*)p(f^*|x^*, w)q(w)df^*dw
  \end{equation}
  Minimising the Kullback–Leibler divergence is equivalent to maximising the \textit{log evidence lower bound},
  \begin{equation}
    KL_{VI} := \int q(w)p(F|X, w) \log p(Y|F) dFdw - KL(q(w)\|p(w))
  \end{equation}
\end{frame}
\begin{frame}{Uncertainties in Bayesian Learning}
  \begin{itemize}
    \item<1-> \textit{Aleatoric} uncertainty: measure the noise inherent in the observations. 
    It is present in the data collection method. It cannot be reduced if more data is collected.
    \item<1-> \textit{Epistemic} uncertainty: represent the uncertainty caused by the model.
    It can be reduced given more data and is often referred to as \textit{model uncertainty}.
  \end{itemize}
  Epistemic uncertainty is modelled by placing a prior distribution over a model’s weights and then trying to capture how much these weights
vary given some data. Aleatoric uncertainty, on the other hand, is modelled by placing a
distribution over the output of the model.
\end{frame}
\begin{frame}{Bayes by Backprop (1)}
  \textit{Bayes by Backprop} is a variational inference method to learn the posterior distribution
  on the weights $w \sim q_\theta(w|\mathcal{D})$ of a neural network from which weights w can be sampled in
  backpropagation.
\end{frame}
\begin{frame}{Bayes by Backprop (2)}
  An approximate distribution $q_\theta(w|\mathcal{D})$ is
  defined that is aimed to be as similar as possible to the true posterior $p(w|\mathcal{D})$, measured by
  the KL divergence. Hence, we define the optimal parameters $\theta^{opt}$ as
  \begin{equation}
    \begin{split}
      \theta^{opt} &= arg \  min_\theta KL [q_\theta(w|\mathcal{D}) \| p(w|\mathcal{D})] \\
      &= arg \  min_\theta KL [q_\theta(w|\mathcal{D}) \| p(w)] \\
      &- \mathbb{E}_{q(w|\theta)}[\log p(\mathcal{D}|w)] + \log p(\mathcal{D})
    \end{split}
  \end{equation}
  where
  \begin{equation}
    KL [q_\theta(w|\mathcal{D}) \| p(w)] = \int q_\theta(w|\mathcal{D}) \log \frac{q_\theta(w|\mathcal{D})}{p(w)}dw.
  \end{equation}
\end{frame}

% -----------------------------------------------------------------------------
\section{HCL}
\begin{frame}{Activation Function}
  We apply the \textit{Softplus} function because we want to ensure that the variance $\alpha \mu^2$ never becomes zero.

  \begin{equation}
    \text{Softplus}(x) = \frac{1}{\beta} \cdot \log \big( 1 + \exp(\beta \cdot x) \big)
  \end{equation}

  where $\beta$ is by default set to 1.
\end{frame}
\begin{frame}{Network Architecture}
  For all conducted experiments, we implement the foregoing description of Bayesian Cnns with
  variational inference in LeNet-5 and AlexNet. 

  \textbf{GitHub repositor:} \url{https://github.com/kumar-shridhar/PyTorch-BayesianCN}
\end{frame}
\begin{frame}{Objective Function}

  \begin{equation}
    \mathcal{F}(\mathcal{D}, \theta) \approx \sum_{i = 1}^n \log q_\theta(w^{(i)}|\mathcal{D}) - \log p(w^{(i)}) - \log p(\mathcal{D}|w^{(i)})
  \end{equation}

  where $n$ is the number of draws.
\end{frame}
\begin{frame}{Variational Posterior}
  \[
    \mathcal{F}(\mathcal{D}, \theta) \approx \sum_{i = 1}^n \log q_\theta(w^{(i)}|\mathcal{D}) - \log p(w^{(i)}) - \log p(\mathcal{D}|w^{(i)})
  \]
  The variational posterior is taken as Gaussian distribution centred around mean $\mu$ and variance as $\sigma^2$
  \begin{equation}
    q_\theta(w^{(i)}|\mathcal{D}) = \prod_i \mathcal{N}(w_i|\mu, \sigma^2)
  \end{equation}
  Then,
  \begin{equation}
    \log(q_\theta(w^{(i)}|\mathcal{D})) = \sum_i \log \mathcal{N}(w_i|\mu, \sigma^2)
  \end{equation}
\end{frame}
\begin{frame}{Prior}
  \[
    \mathcal{F}(\mathcal{D}, \theta) \approx \sum_{i = 1}^n \log q_\theta(w^{(i)}|\mathcal{D}) - \log p(w^{(i)}) - \log p(\mathcal{D}|w^{(i)})
  \]
  Define the prior over the weights as a product of individual Gaussians:
  \begin{equation}
    p(w^{(i)}) = \prod_i \mathcal{N}(w_i|0, \sigma_p^2)
  \end{equation}
  Then,
  \begin{equation}
    \log(p(w^{(i)})) = \sum_i \log \mathcal{N}(w_i|0, \sigma_p^2)
  \end{equation}
\end{frame}
\begin{frame}{Likelihood}
  \[
    \mathcal{F}(\mathcal{D}, \theta) \approx \sum_{i = 1}^n \log q_\theta(w^{(i)}|\mathcal{D}) - \log p(w^{(i)}) - \log p(\mathcal{D}|w^{(i)})
  \]
  $\log p(\mathcal{D}|w^{(i)})$ is the likelihood term and is computed using the softmax function.  
\end{frame}
\begin{frame}{Parameter Initialization}
  \begin{itemize}
    \item We express variance $\sigma$ as $\sigma_i = \textit{softplus}(\rho_i)$ where $\rho$ is an unconstrained parameter.
    \item We take the Gaussian distribution and initialize mean $\mu$ as 0 and variance $\sigma$ (and hence $\rho$) randomly. 
    \item We observed mean centred around 0 and a variance starting with a big number and gradually decreasing over time.
    \item We perform gradient descent over $\theta = (\mu, \rho)$, and individual weight $w_i \sim \mathcal{N}(w_i|\mu_i, \sigma_i)$.
  \end{itemize}
\end{frame}
\begin{frame}{Optimizer}
  For all our tasks, we take Adam optimizer to optimize the parameters. We also perform
  the local reparameterization trick as mentioned in the previous section and take the gradient
  of the combined loss function with respect to the variational parameters $(\mu, \rho)$.
\end{frame}
\begin{frame}{Model Pruning}
  \begin{itemize}
    \item Apply an L1 norm over the network and for all the weights value as zero or below a defined threshold are removed and the model is pruned.
    \item Reduce the size of the network to half (AlexNet and LeNet- 5) by reducing the number of filters to half.
  \end{itemize}
\end{frame}

% -----------------------------------------------------------------------------
\section{Experiments}

% -----------------------------------------------------------------------------
\section{Discussion}
\begin{frame}
  
\end{frame}

% -----------------------------------------------------------------------------
\end{document}
